% chapter3.tex
%==========================================================================
\chapter{TITLE OF CHAPTER 3} % En Mayusculas (In Caps)

\section{Just some about formulas}

Use of \fn{equation*}:

\begin{equation*}
a=b
\end{equation*}

Use of \fn{equation}:

\begin{equation}
a=b
\end{equation}

Use of \fn{split} and \fn{equation}:

\begin{equation}\label{xx}
\begin{split}
a& =b+c-d\\
& \quad +e-f\\
& =g+h\\
& =i
\end{split}
\end{equation}

Use of \fn{multline}:

\begin{multline}
a+b+c+d+e+f+b+c+d+e+f+b+c+d+e+f\\
+b+c+d+e+f+b+c+d+e+f+i+j+k+l+m+n
\end{multline}

Use of \fn{gather}:

\begin{gather}
a_1=b_1+c_1\\
a_2=b_2+c_2-d_2+e_2 \label{eq:D}
\end{gather}

Use of \fn{align}:

\begin{align}
a_1& =b_1+c_1\\
a_2& =b_2+c_2-d_2+e_2
\end{align}

Other uses for \fn{align}:

\begin{align}
a_{11}& =b_{11}&
a_{12}& =b_{12}\\
a_{21}& =b_{21}&
a_{22}& =b_{22}+c_{22}
\end{align}

Use of \fn{flalign*}:

\begin{flalign*}
a_{11}& =b_{11}&
a_{12}& =b_{12}\\
a_{21}& =b_{21}&
a_{22}& =b_{22}+c_{22}
\end{flalign*}

Use of \cn{equation} and \cn{split}:

\begin{equation}\label{e:barwq}\begin{split}
H_c&=\frac{1}{2n} \sum^n_{l=0}(-1)^{l}(n-{l})^{p-2}
\sum_{l _1+\dots+ l _p=l}\prod^p_{i=1} \binom{n_i}{l _i}\\
&\quad\cdot[(n-l )-(n_i-l _i)]^{n_i-l _i}\cdot \Bigl[(n-l
)^2-\sum^p_{j=1}(n_i-l _i)^2\Bigr].
\end{split}\end{equation}

Use of \cn{align} to align textual annotations:

\begin{align}
x& = y_1-y_2+y_3-y_5+y_8-\dots
&& \text{by \eqref{eq:C}}\\
& = y \circ y^* && \text{by \eqref{eq:D}}\\
& = y(0) y && \text {by Axiom 1.}
\end{align}

Use of \cn{aligned} to control placement of inner alignments:

\begin{equation*}
\begin{aligned}
\alpha&=\alpha\alpha\\
\beta&=\beta\beta\beta\beta\beta\\
\gamma&=\gamma
\end{aligned}
\qquad\text{versus}\qquad
\begin{aligned}[t]
\delta&=\delta\delta\\
\eta&=\eta\eta\eta\eta\eta\eta\\
\varphi&=\varphi
\end{aligned}
\end{equation*}

``Cases" constructions:

\begin{equation}\label{eq:C}
P_{r-j}=
\begin{cases}
0& \text{if $r-j$ is odd},\\
r!\,(-1)^{(r-j)/2}& \text{if $r-j$ is even}.
\end{cases}
\end{equation}

Use of \cn{smash} and \cn{vphantom} to control vertical size:

\newcommand\ip[2]{\langle #1 \/ | #2 \/\rangle}
\newcommand\Ip[2]{\left\langle{\arraycolsep=0pt %
\begin{array}{c|c}#1\/\,&\,#2\/\end{array}}%
\right\rangle}
\newcommand\Ipd[2]{\left\langle{\arraycolsep=0pt %
\begin{array}{c|c}\displaystyle#1\/\,&\,\displaystyle#2\/\end{array}}%
\right\rangle}
\newcommand\conj[1]{\overline{#1}}
\begin{align*}
\Ipd{ u\: }{\: \smash{\sum_{i=1}^n F(e_i,v) e_i }\vphantom{\sum}}
&= \sum_{i=1}^n F(e_i,v) \ip{ u }{ e_i } \\
&= \sum_{i=1}^n \ip{ u }{ e_i } F(e_i,v) \\
&= \sum_{i=1}^n \conj{\ip{ e_i }{ u }} F(e_i,v) \\
&= F \biggl( \sum_{i=1}^n \ip{ e_i }{ u } e_i,v\biggr) = F( u, v
),
\end{align*}